\pagebreak
\section{付加形式の敬語}

\subsection{前につける形}

\begin{longtblr}{
    cells = {c,m,font = \small},
	colspec = {|X|X|X|X|X|X|X|X|},
    hline{1-2,Z} = {solid},
    %hline{4,6,7} = {dotted, rightpos = -0.1, leftpos = -0.1},
    %hline{8} = {1-2}{dotted, rightpos = -0.1, leftpos = -0.1},
    %hline{9} = {3-4}{dotted, rightpos = -0.1, leftpos = -0.1},
    %column{6} = {cmd = \arabic{rownum}},%标注行号,方便画线,仅在第一行后增添“&”即可。注意要在colsep里新设置一栏。
    row{1} = {font = \gtfamily\sgseries},
    column{2,8} = {cmd = \jidori{3\zw}},
    column{1,3,5,7} = {font = \gtfamily\sgseries}
}
\SetCell[c=2]{c}尊&&\SetCell[c=2]{c}謙(Ⅰ)&&\SetCell[c=2]{c}謙(Ⅱ)&&\SetCell[c=2]{c}丁・美& \\
\SetCell[r=2]{c}お~&お考え&&お返事&&&&お料理 \\
&お電話&&お電話&&&&お酒 \\
\SetCell[r=2]{c}\ruby{御}{ご}~&ご住所&&ご返事&&&&\SetCell[r=2]{c}ご馳走 \\
&ご挨拶&&ご挨拶&&&& \\
\ruby{御}{おん}~&御礼&&&&&&お熱い \\
\ruby{御}{ぎょ}~&御物&\SetCell[r=2]{c}\ruby{相}{あい}~&\footnotesize 相すみ\linebreak ません&&&& \\
\SetCell[r=2]{c}\ruby{御}{み}~&み\ruby{仏}{ほとけ}&&\footnotesize 相成り\linebreak\jidori{3\zw}{ます}&&&& \\
&お\ruby{神|籤}{み|くじ}&&&&& \\
\end{longtblr}

\subsection{接頭語ほか}

\begin{longtblr}{
    cells = {c,m,font = \small},
	colspec = {|X|X|X|X|X|X|X|X|},
    hline{1-2,Z} = {solid},
    %hline{3,5,7,8,10} = {dotted, rightpos = -0.1, leftpos = -0.1},
    %column{6} = {cmd = \arabic{rownum}},%标注行号,方便画线,仅在第一行后增添“&”即可。注意要在colsep里新设置一栏。
    row{1} = {font = \gtfamily\sgseries},
    column{2,6} = {cmd = \jidori{3\zw}},
    column{1,3,5,7} = {font = \gtfamily\sgseries}
}
\SetCell[c=2]{c}尊&&\SetCell[c=2]{c}謙(Ⅰ)&&\SetCell[c=2]{c}謙(Ⅱ)&&\SetCell[c=2]{c}丁・美& \\
\ruby{貴}{き}~&貴社&&&\ruby{弊}{へい}~&弊社&& \\
\SetCell[r=2]{c}\ruby{芳}{ほう}~&芳名&&&\SetCell[r=2]{c}\ruby{拙}{せつ}~&\SetCell[r=2]{c}\ruby{拙}{せっ}宅&& \\
&ご芳名&&&&&& \\
\SetCell[r=2]{c}\ruby{令}{れい}~&令\ruby{兄}{けい}&&&&&& \\
&ご令兄&&&&&& \\
\ruby{大}{たい}~&大兄&&&\ruby{小}{しょう}~&小職&& \\
\SetCell[r=2]{c}\ruby{尊}{そん}~&尊\ruby{父}{ぷ}&&&\SetCell[r=2]{c}\ruby{愚}{ぐ}~&\SetCell[r=2]{c}愚\ruby{妻}{さい}&& \\
&ご尊父&&&&&& \\
\ruby{玉}{ぎょく}~&\ruby{玉}{ぎょっ}稿&&&\ruby{粗}{そ}~&\ruby{粗|品}{そ|しな}&& \\
\end{longtblr}

\pagebreak
\vspace*{-2\zw}
\subsection{後につける形}
\begin{longtblr}{
    cells = {c,m,font = \small},
	colspec = {|X|X|X|X|X|X|X|X|},
    hline{1-2,Z} = {solid},
    %hline{3-Y} = {dotted, rightpos = -0.1, leftpos = -0.1},
    %column{6} = {cmd = \arabic{rownum}},%标注行号,方便画线,仅在第一行后增添“&”即可。注意要在colsep里新设置一栏。
    row{1} = {font = \gtfamily\sgseries},
    column{2,4} = {cmd = \jidori{3\zw}},
    column{1,3,5,7} = {font = \gtfamily\sgseries}
}
\SetCell[c=2]{c}尊&&\SetCell[c=2]{c}謙(Ⅰ)&&\SetCell[c=2]{c}謙(Ⅱ)&&\SetCell[c=2]{c}丁・美& \\
~さん&\SetCell{cmd = {}}鈴木さん&~ども&私ども&&&& \\
~\ruby{様}{さま}&鈴木様&~め&\ruby{倅}{せがれ}め&&&& \\
~\ruby{君}{くん}&鈴木君&&&&&~君& \\
~\ruby{上}{うえ}&父上&&&&&& \\
~\ruby{氏}{し}&鈴木氏&~\ruby{儀}{ぎ}&一郎儀&&&& \\
~\ruby{殿}{どの}&鈴木殿&&&&& \\
~\ruby{方}{がた}&皆様方&&&&&& \\
\end{longtblr}

\subsection{接尾語ほか}
\begin{longtblr}{
    cells = {c,m,font = \small},
	colspec = {|X|X|X|X|},
    hline{1-2,Z} = {solid},
    %hline{3-Y} = {dotted, rightpos = -0.1, leftpos = -0.1},
    %column{6} = {cmd = \arabic{rownum}},%标注行号,方便画线,仅在第一行后增添“&”即可。注意要在colsep里新设置一栏。
    row{1} = {font = \gtfamily\sgseries},
    column{1,3} = {font = \gtfamily\sgseries}
}
\SetCell[c=2]{c}尊&&\SetCell[c=2]{c}謙(Ⅰ)& \\
~れる&読まれる&& \\
~られる&教えられる&&\\
\SetCell[r=2]{c}~てくださる&\SetCell[r=2]{c}見てくださる&~ていただく&教えていただく\\
&&\SetCell{font = \gtfamily\sgseries\small}~させていただく&休ませていただく \\
~てごらん&見てごらん&&\\
~なさる&報告なさる&&\\
\end{longtblr}

\subsection{前後につける形}
\begin{longtblr}{
    cells = {c,m,font = \small},
	colspec = {|X|X|X|X|},
    hline{1-2,Z} = {solid},
    %hline{16} = {dotted, rightpos = -0.1, leftpos = -0.1},
    %column{6} = {cmd = \arabic{rownum}},%标注行号,方便画线,仅在第一行后增添“&”即可。注意要在colsep里新设置一栏。
    row{1} = {font = \gtfamily\sgseries},
    column{1,3} = {font = \gtfamily\sgseries},
    rowhead = 1
}
\SetCell[c=2]{c}尊&&\SetCell[c=2]{c}謙(Ⅰ)& \\
お~さん&お父さん&お~申す&お連れ申す \\
お~様&お父様&お~申し上げる&お誘い申し上げる \\
\ruby{御}{おん}~上&御母上&ご~申し上げる&ご説明申し上げる \\
\ruby{御}{おん}~上様&御母上様&& \\
ご~さん&ご隠居さん&& \\
ご~様&ご両親様&& \\
お~になる&お帰りになる&お~する&お呼びする \\
ご~になる&ご心配になる&ご~する&ご紹介する \\
お~なさる&お書きなさる&お~いたす&お見せいたす \\
ご~なさる&ご案内なさる&ご~いたす&ご案内いたす \\
お~くださる&お教えくださる&お~いただく&お褒めいただく \\
ご~くださる&ご説明くださる&ご~いただく&ご説明いただく \\
お~あそばす&お書きあそばす&& \\
ご~あそばす&ご出席あそばす&& \\
\end{longtblr}

\begin{longtblr}{
    cells = {c,m,font = \small},
	colspec = {|X|X|X|X|},
    hline{1,Z} = {solid},
    %hline{16} = {dotted, rightpos = -0.1, leftpos = -0.1},
    %column{6} = {cmd = \arabic{rownum}},%标注行号,方便画线,仅在第一行后增添“&”即可。注意要在colsep里新设置一栏。
    column{1,3} = {font = \gtfamily\sgseries}
}
お~です&お帰りです&お~ねがう&お伝え願う \\
ご~です&ご出張です&ご~ねがう&ご説明願う \\
\end{longtblr}